\documentclass{article}
\usepackage{amssymb,amsmath}
\title{Seven Sketches in Compositionality – Exercises}
\author{Adam Catto}
\date{}
\begin{document}
\maketitle
\section{Chapter 1 –– Generative Effects: Orders and Galois Connections}
\textbf{(1.1)}
\begin{itemize}
\item[(a)] 
\textit{order-preserving}  $f: x\mapsto x+1$\\
\textit{non-order-preserving}  $f: x\mapsto -x$
\item[(b)] \textit{metric-preserving} $x\mapsto x+2$\\
\textit{non-metric-preserving} $x\mapsto 2x$
\item[(c)] \textit{addition-preserving} $x\mapsto x$\\
\textit{non-addition-preserving} $x\mapsto 2x$
\end{itemize}\bigskip
\textbf{(1.2)}\\
 \\
 Circle 21, Circle the rest, box around the whole thing. i.e.
 $$\{ \{ 21 \}, \{ 11,12,13,22,23 \} \}$$\bigskip
\textbf{(1.6)}
\begin{enumerate}
	\item True
	\item False
	\item True
\end{enumerate}\bigskip
\textbf{(1.7)}
\begin{enumerate}
	\item $\{ \}, \{ 1\}, \{ 2\}, \{ 3\}, \{ 1,2\}, \{1,3\}, \{ 2,3\}, \{ 1,2,3\}$
	\item $\{ 1\} \cup \{ 1,3\} = \{ 1,3\}$
	\item $(h,1),(h,2),(h,3),(1,1),(1,2),(1,3)$
	\item $(h,1),(1,1),(1,2),(2,2),(3,2)$
	\item $A\cup B = \{ h,1,2,3\}$
\end{enumerate}\bigskip
\textbf{1.11}
\begin{enumerate}
	\item If there were more than one $p'\in P'$ such that $A_p=A'_{p'}$ (i.e. $p'_1$ and $p'_2$ such that $A_p=A'_{p'_1}$ and $A_p=A'_{p'_2}$), then $p'_1\neq p'_2$, so necessarily $A'_{p'_1} \cap A'_{p'_2}=\emptyset$. But then since $A'_{p'_1}=A_p=A'_{p'_2}$, then $A'_{p'_1}=A'_{p'_2}$, thus $A'_{p'_1}\cap A'_{p'_2} = A_p\cap A_p = A_p\neq \emptyset$, by definition of partition, therefore there cannot be more than one $p'\in P'$ such that $A_p=A'_{p'}$. $\hfill \blacksquare$
	\item Since there exists a $p'\in P'$ such that $A_p=A'_{p'}$, and by 1.11.1, there is at most one such $p'$, it follows that there is a bijection between these $p\in P$ and $p'\in P'$, thus $\forall p'\in P'$ there exists a $p\in P$ such that $A_p=A'_{p'}$. $\hfill \blacksquare$
\end{enumerate}\bigskip
\textbf{1.12}
$$ (11,11),(12,12),(13,13),(21,21),(22,22),(23,23),(11,12),(12,11),(22,23),(23,22) $$
\textbf{1.15}
\begin{enumerate}
	\item Each $A_p$ is $(\sim )$-connected, therefore they are nonempty.$\hfill\blacksquare$
	\item If $A_p\cap A_q\neq\emptyset$, then for any $x\in A_p\cap A_q$, it follows  that $x\in A_p$ and $x\in A_q$. Thus for any $x_p\in A_p$, we have $x_p\sim x$, and under $(\sim )$-closure and $x\in A_q$, it follows that $x_p\in A_q$, and the same for $x_q\in A_q$. Therefore, $A_p=A_q \vDash p=q$, which violates the contextual assertion that $p\neq q$, therefore $p\neq q\implies A_p\cap A_q=\emptyset$.$\hfill\blacksquare$
	\item Each $A_p$ is nonempty and $\sim$ is reflexive, i.e. we have at least for each $x$ that $x\sim x$, therefore we have at least that $A$ is the union of singleton sets that cover $A$, therefore $A=\bigcup_{p\in P} A_p$.$\hfill\blacksquare$
\end{enumerate}\bigskip
\textbf{1.19}
\begin{enumerate}
	\item $f:\mathbb{Z}\rightarrow\mathbb{R}: x\mapsto x+1$
	\item $f:\mathbb{Q}\rightarrow\mathbb{Z}: x\mapsto x-\frac{x}{10}$
	\item (1): yes (2): no (3): no (4): yes 
	\item (1): neither (2): top dot's targets are not unique, bottom dot has no target (3): top dot has no target (partial function?) (4) bijective
\end{enumerate}\bigskip
\textbf{1.20}\\
 \\
By function definition, each $a\in A$ has a unique $y\in\emptyset$ such that $(a,y)\in f$. But there are no $y\in\emptyset$, therefore there cannot exist any such $a\in A$, therefore $A$ is empty.$\hfill\blacksquare$
\end{document}