\documentclass{article}
\usepackage{amssymb}
\title{Seven Sketches in Compositionality – Exercises}
\author{Adam Catto}
\date{}
\begin{document}
\maketitle
\section{Chapter 1 –– Generative Effects: Orders and Galois Connections}
\textbf{(1.1)}
\begin{itemize}
\item[(a)] 
\textit{order-preserving}  $f: x\mapsto x+1$\\
\textit{non-order-preserving}  $f: x\mapsto -x$
\item[(b)] \textit{metric-preserving} $x\mapsto x+2$\\
\textit{non-metric-preserving} $x\mapsto 2x$
\item[(c)] \textit{addition-preserving} $x\mapsto x$\\
\textit{non-addition-preserving} $x\mapsto 2x$
\end{itemize}\bigskip
\textbf{(1.2)}\\
 \\
 Circle 21, Circle the rest, box around the whole thing. i.e.
 $$\{ \{ 21 \}, \{ 11,12,13,22,23 \} \}$$\bigskip
\textbf{(1.6)}
\begin{enumerate}
	\item True
	\item False
	\item True
\end{enumerate}\bigskip
\textbf{(1.7)}
\begin{enumerate}
	\item $\{ \}, \{ 1\}, \{ 2\}, \{ 3\}, \{ 1,2\}, \{1,3\}, \{ 2,3\}, \{ 1,2,3\}$
	\item $\{ 1\} \cup \{ 1,3\} = \{ 1,3\}$
	\item $(h,1),(h,2),(h,3),(1,1),(1,2),(1,3)$
	\item $(h,1),(1,1),(1,2),(2,2),(3,2)$
	\item $A\cup B = \{ h,1,2,3\}$
\end{enumerate}\bigskip
\textbf{1.11}
\begin{enumerate}
	\item If there were more than one $p'\in P'$ such that $A_p=A'_{p'}$ (i.e. $p'_1$ and $p'_2$ such that $A_p=A'_{p'_1}$ and $A_p=A'_{p'_2}$), then $p'_1\neq p'_2$, so necessarily $A'_{p'_1} \cap A'_{p'_2}=\emptyset$. But then since $A'_{p'_1}=A_p=A'_{p'_2}$, then $A'_{p'_1}=A'_{p'_2}$, thus $A'_{p'_1}\cap A'_{p'_2} = A_p\cap A_p = A_p\neq \emptyset$, by definition of partition, therefore there cannot be more than one $p'\in P'$ such that $A_p=A'_{p'}$. $\hfill \blacksquare$
	\item Since there exists a $p'\in P'$ such that $A_p=A'_{p'}$, and by 1.11.1, there is at most one such $p'$, it follows that there is a bijection between these $p\in P$ and $p'\in P'$, thus for each $p'\in P'$ there exists a $p\in P$ such that $A_p=A'_{p'}$. $\hfill \blacksquare$
\end{enumerate}
\end{document}